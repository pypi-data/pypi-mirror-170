@c                                           -*- Texinfo -*-
@node Syscalls
@chapter System Calls

@cindex linking the C library
The C subroutine library depends on a handful of subroutine calls for
operating system services.  If you use the C library on a system that
complies with the POSIX.1 standard (also known as IEEE 1003.1), most of
these subroutines are supplied with your operating system.

If some of these subroutines are not provided with your system---in
the extreme case, if you are developing software for a ``bare board''
system, without an OS---you will at least need to provide do-nothing
stubs (or subroutines with minimal functionality) to allow your
programs to link with the subroutines in @code{libc.a}.

@menu
* Stubs::		Definitions for OS interface
* Reentrant Syscalls::	Reentrant covers for OS subroutines
@end menu

@node Stubs
@section Definitions for OS interface
@cindex stubs

@cindex subroutines for OS interface
@cindex OS interface subroutines
This is the complete set of system definitions (primarily subroutines)
required; the examples shown implement the minimal functionality
required to allow @code{libc} to link, and fail gracefully where OS
services are not available.  

Graceful failure is permitted by returning an error code.  A minor
complication arises here: the C library must be compatible with
development environments that supply fully functional versions of these
subroutines.  Such environments usually return error codes in a global
@code{errno}.  However, the Red Hat newlib C library provides a @emph{macro}
definition for @code{errno} in the header file @file{errno.h}, as part
of its support for reentrant routines (@pxref{Reentrancy,,Reentrancy}).

@cindex @code{errno} global vs macro
The bridge between these two interpretations of @code{errno} is
straightforward: the C library routines with OS interface calls
capture the @code{errno} values returned globally, and record them in
the appropriate field of the reentrancy structure (so that you can query
them using the @code{errno} macro from @file{errno.h}).

This mechanism becomes visible when you write stub routines for OS
interfaces.   You must include @file{errno.h}, then disable the macro,
like this:

@example
#include <errno.h>
#undef errno
extern int errno;
@end example

@noindent
The examples in this chapter include this treatment of @code{errno}.

@ftable @code
@item _exit
Exit a program without cleaning up files.  If your system doesn't
provide this, it is best to avoid linking with subroutines that require
it (@code{exit}, @code{system}).

@item close
Close a file.  Minimal implementation:

@example
int close(int file) @{
  return -1;
@}
@end example

@item environ
A pointer to a list of environment variables and their values.  For a
minimal environment, this empty list is adequate:

@example
char *__env[1] = @{ 0 @};
char **environ = __env;
@end example

@item execve
Transfer control to a new process.  Minimal implementation (for a system
without processes):

@example
#include <errno.h>
#undef errno
extern int errno;
int execve(char *name, char **argv, char **env) @{
  errno = ENOMEM;
  return -1;
@}
@end example

@item fork
Create a new process.  Minimal implementation (for a system without processes):

@example
#include <errno.h>
#undef errno
extern int errno;
int fork(void) @{
  errno = EAGAIN;
  return -1;
@}
@end example

@item fstat
Status of an open file.  For consistency with other minimal
implementations in these examples, all files are regarded as character
special devices.  The @file{sys/stat.h} header file required is
distributed in the @file{include} subdirectory for this C library.

@example
#include <sys/stat.h>
int fstat(int file, struct stat *st) @{
  st->st_mode = S_IFCHR;
  return 0;
@}
@end example

@item getpid
Process-ID; this is sometimes used to generate strings unlikely to
conflict with other processes.  Minimal implementation, for a system
without processes:

@example
int getpid(void) @{
  return 1;
@}
@end example

@item isatty
Query whether output stream is a terminal.   For consistency with the
other minimal implementations, which only support output to
@code{stdout}, this minimal implementation is suggested:

@example
int isatty(int file) @{
  return 1;
@}
@end example

@item kill
Send a signal.  Minimal implementation:

@example
#include <errno.h>
#undef errno
extern int errno;
int kill(int pid, int sig) @{
  errno = EINVAL;
  return -1;
@}
@end example

@item link
Establish a new name for an existing file.  Minimal implementation:

@example
#include <errno.h>
#undef errno
extern int errno;
int link(char *old, char *new) @{
  errno = EMLINK;
  return -1;
@}
@end example

@item lseek
Set position in a file.  Minimal implementation:

@example
int lseek(int file, int ptr, int dir) @{
  return 0;
@}
@end example

@item open
Open a file.  Minimal implementation:

@example
int open(const char *name, int flags, int mode) @{
  return -1;
@}
@end example

@item read
Read from a file.  Minimal implementation:

@example
int read(int file, char *ptr, int len) @{
  return 0;
@}
@end example

@item sbrk
Increase program data space.  As @code{malloc} and related functions
depend on this, it is useful to have a working implementation.  The
following suffices for a standalone system; it exploits the symbol
@code{_end} automatically defined by the GNU linker.

@example
@group
caddr_t sbrk(int incr) @{
  extern char _end;		/* @r{Defined by the linker} */
  static char *heap_end;
  char *prev_heap_end;
 
  if (heap_end == 0) @{
    heap_end = &_end;
  @}
  prev_heap_end = heap_end;
  if (heap_end + incr > stack_ptr) @{
    write (1, "Heap and stack collision\n", 25);
    abort ();
  @}

  heap_end += incr;
  return (caddr_t) prev_heap_end;
@}
@end group
@end example

@item stat
Status of a file (by name).  Minimal implementation:

@example
int stat(char *file, struct stat *st) @{
  st->st_mode = S_IFCHR;
  return 0;
@}
@end example

@item times
Timing information for current process.  Minimal implementation:

@example
int times(struct tms *buf) @{
  return -1;
@}
@end example

@item unlink
Remove a file's directory entry.  Minimal implementation:

@example
#include <errno.h>
#undef errno
extern int errno;
int unlink(char *name) @{
  errno = ENOENT;
  return -1; 
@}
@end example

@item wait
Wait for a child process.  Minimal implementation:
@example
#include <errno.h>
#undef errno
extern int errno;
int wait(int *status) @{
  errno = ECHILD;
  return -1;
@}
@end example

@item write
Write to a file.  @file{libc} subroutines will use this
system routine for output to all files, @emph{including}
@code{stdout}---so if you need to generate any output, for example to a
serial port for debugging, you should make your minimal @code{write}
capable of doing this.  The following minimal implementation is an
incomplete example; it relies on a @code{outbyte} subroutine (not
shown; typically, you must write this in assembler from examples
provided by your hardware manufacturer) to actually perform the output.

@example
@group
int write(int file, char *ptr, int len) @{
  int todo;

  for (todo = 0; todo < len; todo++) @{
    outbyte (*ptr++);
  @}
  return len;
@}
@end group
@end example

@end ftable

@page
@node Reentrant Syscalls
@section Reentrant covers for OS subroutines

Since the system subroutines are used by other library routines that
require reentrancy, @file{libc.a} provides cover routines (for example,
the reentrant version of @code{fork} is @code{_fork_r}).  These cover
routines are consistent with the other reentrant subroutines in this
library, and achieve reentrancy by using a reserved global data block
(@pxref{Reentrancy,,Reentrancy}).

@menu
* _close_r::	Reentrant version of close
* _execve_r::	Reentrant version of execve
* _fork_r::	Reentrant version of fork
@ifset STDIO64
* _fstat64_r::	Reentrant version of fstat64
@end ifset
* _fstat_r::	Reentrant version of fstat
* _getpid_r::	Reentrant version of getpid
* _kill_r::	Reentrant version of kill
* _link_r::	Reentrant version of link
@ifset STDIO64
* _lseek64_r::	Reentrant version of lseek64
@end ifset
* _lseek_r::	Reentrant version of lseek
@ifset STDIO64
* _open64_r::	Reentrant version of open64
@end ifset
* _open_r::	Reentrant version of open
* _read_r::	Reentrant version of read
* _sbrk_r::	Reentrant version of sbrk
@ifset STDIO64
* _stat64_r::	Reentrant version of stat64
@end ifset
* _stat_r::	Reentrant version of stat
* _times_r::	Reentrant version of times
* _unlink_r::	Reentrant version of unlink
* _wait_r::	Reentrant version of wait
* _write_r::	Reentrant version of write
@end menu

@lowersections
@page
@include reent/closer.def

@page
@include reent/execr.def

@ifset STDIO64
@page
@include reent/fstat64r.def
@end ifset

@page
@include reent/fstatr.def

@page
@include reent/linkr.def

@ifset STDIO64
@page
@include reent/lseek64r.def
@end ifset

@page
@include reent/lseekr.def

@ifset STDIO64
@page
@include reent/open64r.def
@end ifset

@page
@include reent/openr.def

@page
@include reent/readr.def

@page
@include reent/sbrkr.def

@page
@include reent/signalr.def

@ifset STDIO64
@page
@include reent/stat64r.def
@end ifset

@page
@include reent/statr.def

@page
@include reent/timesr.def

@page
@include reent/unlinkr.def

@page
@include reent/writer.def
@raisesections
