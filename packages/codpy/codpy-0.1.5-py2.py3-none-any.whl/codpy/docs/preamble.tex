\usepackage{booktabs}
\usepackage{amsthm}
\makeatletter
\def\thm@space@setup{%
  \thm@preskip=8pt plus 2pt minus 4pt
  \thm@postskip=\thm@preskip
}
\makeatother
\usepackage{graphics}
\usepackage{amsmath, amsfonts, amsthm, amssymb, amscd,a4wide}
\usepackage{hyperref}
\usepackage{url}
\usepackage{float}
\usepackage{epstopdf}
\usepackage{subcaption}
\let\oldmarginpar\marginpar
\renewcommand\marginpar[1]{\-\oldmarginpar[\raggedleft\footnotesize #1]%
{\raggedright\footnotesize #1}}

%\usepackage[linesnumbered,ruled]{algorithm2e}

%%

% ces lignes permet de ne pas mettre en couleurs les liens hypertextes dans le fichier PDF
% cependant il est toujours possible de cliquer dessus
%

\usepackage{xcolor} %package pour les couleurs
\usepackage{tikz} % package principal TikZ
\usetikzlibrary{arrows} %librairieoptionnelle PGF
\usepackage{adjustbox}

\usepackage{slashed}
\usepackage{amsmath}
\usepackage{xcolor}
\usepackage{hyperref}
\hypersetup{linktoc = all}                % hyperref settings
%\hypersetup{pdfborderstyle={/S/U/W 0.5}}  % hyperref settings
\hypersetup{hidelinks}
\hypersetup{bookmarksnumbered}
\pdfstringdefDisableCommands{%
  \def\({}%
  \def\){}%
  \def\\{}%
  \def\infty{\042\036}%
  \def\Tr{Tr }%
}
%%%%%%%%% 
\newtheorem{definition}{Definition}[section]
\newtheorem{corollary}[definition]{Corollary}
\newtheorem{lemma}[definition]{Lemma}
\newtheorem{theorem}[definition]{Theorem}
\newtheorem{proposition}[definition]{Proposition}
\newtheorem{theoremdefinition}[definition]{Theorem and Definition}
\newtheorem{remark}[definition]{Remark}
\newtheorem{conjecture}{Conjecture}
\newtheorem{example}[definition]{Example}
\newtheorem{HP}{Highlighted point}
% 
\numberwithin{equation}{section}
\newcommand\norm[1]{\left\lVert#1\right\rVert}
\newcommand \be   {\begin{equation}}
\newcommand \bel {\begin{equation}\label}
\newcommand \ee   {\end{equation}}
\newcommand \dist {{\mbox{\em dist }}}
\newcommand \sgn {{\text{sgn }}}
\newcommand \meas {{\text{meas }}}
\newcommand \supp {{\text{supp }}}
\newcommand \Id   {{\text{Id}}}
\newcommand \smin {s^{\text{min}}}
\newcommand \smax {s^{\text{max}}}
\newcommand \lmin {{\lam^{\text{min}}}}
\newcommand \lmax {{\lam^{\text{max}}}}
\newcommand \RR    {\mathbb{R}}
\newcommand \NN    {\mathbb{N}}
\newcommand \ZZ    {\mathbb{Z}}
\newcommand \QQ    {\mathbb{Q}}
\newcommand \PP    {\mathbb{P}}
\newcommand \EE    {\mathbb{E}}
\newcommand \Rp    {\mathbb{R}^\plus }
\newcommand \RRR    {\mathbf{R}}
\newcommand \SSS    {\mathbf{S}}
\newcommand \Scal    {\mathcal{S}}
\newcommand \Pcal    {\mathcal{P}}
\newcommand \Tbar {\overline T}
\newcommand \Acal {\mathcal A}
\newcommand \Bcal {\mathcal B}
\newcommand \Ccal    {\mathcal{C}}
\newcommand \Mcal    {\mathcal{M}}
\newcommand \Lcal    {\mathcal{L}}
\newcommand \Jcal    {\mathcal{J}}
\newcommand \Kcal    {\mathcal{K}}
\newcommand \Wbf {\mathbf W}
\newcommand \Hcal    {\mathcal{H}}
\newcommand \Tcal    {\mathcal{T}}
\newcommand \lam   {\lambda}
\newcommand \sig   {\sigma}
\newcommand \gam   {\gamma}
\newcommand \ubar   {\overline u}
\newcommand \HH    {\mathcal{H}}
\newcommand \CC    {\mathcal{C}}
\newcommand \Ncal    {\mathcal{N}}
\newcommand \DDD    {\mathcal{D}}
\newcommand \RN    {{\RR^N}}
\newcommand \eps   {\epsilon}
\newcommand \Lam   {\Lambda}
\newcommand \BB    {{\mathcal B}}
\newcommand \WW    {{\mathcal W}}
\newcommand \MM    {{M}}
\newcommand \AAA    {{\mathcal A}}
\newcommand \JJ    {{\mathcal J}}
\newcommand \II    {{\mathcal I}}
\newcommand \LLL    {{\mathbf L}}
\newcommand \VVV    {{\mathbf V}}
\newcommand \QQQ    {{\mathbf Q}}
\newcommand \Rd    {{\mathbb{R}^d}}
\newcommand \CCD    {{\mathbb{C}^D}}
%
\newcommand \del   {\partial}
\newcommand \blam  {{\underline\lambda}}
\newcommand \lamb  {{\overline\lambda}}
\newcommand \Bzero    {{\mathcal{B}_{\delta_0}}}
\newcommand \Bone    {{\mathcal{B}_{\delta_1}}}
\newcommand \Btwo    {{\mathcal{B}_{\delta_2}}}
\newcommand \la         \langle
\newcommand \ra     \rangle
\newcommand \ab     {\overline a}
\newcommand \mmm  {p}

\newcommand \Ybf {\mathbf Y} 
\newcommand \Sbf {\mathbf S} 
\newcommand \hbf {\mathbf h} 

\newcommand \Sbar {\overline S}
\newcommand \Aund {\underline{\Acal}}
\newcommand \Aove {\overline{\Acal}}

\newcommand \plus {+}

\newcommand \RD {{\mathbb R}^D}


\usepackage{mathrsfs}

%*************************************************************************************
\usepackage{authblk}


\setcounter{Maxaffil}{0}
\renewcommand\Affilfont{\itshape\small}



 
 
\tikzstyle{startstop} = [rectangle, rounded corners, minimum width=3cm, minimum height=1cm,text centered, draw=black ]
\tikzstyle{io} = [rectangle, rounded corners, minimum width=3cm, minimum height=1cm,text centered, draw=black] 
%\tikzstyle{io} = [trapezium, trapezium left angle=70, trapezium right angle=110, minimum width=3cm, minimum height=1cm, text centered, draw=black, fill=blue!30]
\tikzstyle{process} = [rectangle, minimum width=3cm, minimum height=1cm, text centered, draw=black, fill=orange!30]
\tikzstyle{decision} = [diamond, minimum width=3cm, minimum height=1cm, text centered, text - white, draw=black, fill=black!30] 
\tikzstyle{arrow} = [thick,->,>=stealth]
\tikzstyle{bbox} = [rectangle, minimum width=3cm, minimum height=1cm, text centered, text = white, draw=black, fill=black]
\tikzstyle{pp} = [rectangle, minimum width=3cm, minimum height=0.5cm, text centered, draw=black] 
\tikzstyle{pp1} = [rectangle, draw=black!50, thick, minimum width=0.5cm, minimum height = 0.5cm]
\tikzstyle{crl} = [circle, draw=black!50, thick, minimum size = 1.5cm]
\tikzstyle{crl1} = [circle, draw=black!50, thick, minimum size = 0.7cm]
\tikzstyle{line} = [draw, -latex']
\newsavebox{\tempbox}

\tikzstyle{block} = [draw, rectangle, 
    minimum height=3em, minimum width=6em]
\tikzstyle{sum} = [draw, fill=blue!20, circle, node distance=1cm]
\tikzstyle{input} = [coordinate]
\tikzstyle{output} = [coordinate]
\tikzstyle{pinstyle} = [pin edge={to-,thin,black}]
