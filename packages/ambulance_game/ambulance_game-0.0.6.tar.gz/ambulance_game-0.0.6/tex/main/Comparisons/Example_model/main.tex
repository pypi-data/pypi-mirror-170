\section{Markov chain VS Simulation}

\subsection{Example model}
Consider the Markov chain paradigm in figure \ref{Model_mini}. 
The illustrated model represents the unrealistically small system 
with a system capacity of five and a buffer capacity of three. 
The hospital in this particular example also has four servers and a threshold of 
three; meaning that every ambulance that arrives in a time that there are three or 
more individuals in the hospital, will proceed to the buffer centre.

\begin{figure}[h]
    \centering
    \section{Abstract}
Emergency departments (EDs) in hospitals are usually under pressure to achieve a 
target amount of time that describes the arrival of patients and the time it takes 
to receive treatment. 
For example in the UK this is often set as 95\% of patients to be treated within 
4 hours. 
There is empirical evidence to suggest that imposing targets in the ED results in 
gaming at the interface of care between the EMS and ED. 
If the ED is busy and a patient is stable in the ambulance, there is little 
incentive for the ED to accept the patient whereby the clock will start ticking 
on the 4 hour target. 
This in turn impacts on the ability of the EMS to respond to emergency calls.

This study explores the impact that this effect may have on an ambulance's 
utilisation and their ability to respond to emergency calls. 
More specifically multiple scenarios are examined where an ambulance service needs 
to distribute patients between neighbouring hospitals. 
The interaction between the hospitals and the ambulance service is defined in a 
game theoretic framework where the ambulance service has to decide how many patients 
to distribute to each hospital in order to minimise the occurrence of this effect. 
The methodology involves the use of a queueing model for each hospital that is used 
to inform the decision process of the ambulance service so as to create a game for 
which the Nash Equilibria can be calculated.

    \caption{Markov chains: number of servers=4} 
    \label{Model_mini}
\end{figure}

In addition to the Markov chain model a simulation model has also been built based 
on the same parameters. 
Comparing the results of the Markov model and the equivalent simulation model the 
resultant plots arose.

The heatmaps in figure \ref{Heatmap_mini} represent the state probabilities for 
the Markov chain model, the simulation model and the difference between the two. 
Each pixel of the heatmap corresponds to the equivalent state of figure \ref{Model_mini} 
and represents the probability of being at that state in any particular moment of time.

It can be observed that both Markov chain and simulation models' state probabilities 
vary from 5\% to 25\% and that states \( (0, 1) \) and \( (0, 2) \) are the most 
visited ones. 
Looking at the differences' heatmap, one may identify that the differences between 
the two are minimal.

\newpage

\begin{figure}[h]
    \includegraphics[width=\linewidth]{Comparisons/Example_model/Heatmap/main.pdf}
    \caption{Heatmaps of Simulation, Markov chains and differences of the two}
    \label{Heatmap_mini}
\end{figure}



